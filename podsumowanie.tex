\documentclass[a4paper]{article}
% UTF-8
\usepackage[T1]{fontenc}
\usepackage[utf8]{inputenc}

% Czcionka, która sensownie wygląda na monitorze.
\usepackage{lmodern}

% Język
\usepackage[polish]{babel}
% \usepackage[english]{babel}

% Rózne przydatne paczki:
% - znaczki matematyczne
\usepackage{amsmath, amsfonts}
% - wcięcie na początku pierwszego akapitu
\usepackage{indentfirst}
% - komenda \url 
\usepackage{hyperref}
% - dołączanie obrazków
\usepackage{graphics}
% - szersza strona
\usepackage[nofoot,hdivide={2cm,*,2cm},vdivide={2cm,*,2cm}]{geometry}
\frenchspacing
% - brak numerów stron
\pagestyle{empty}

% dane autora
\author{Karol Bednarek, Kinga Błaszkiewicz, Jakub Chytroń, Jerzy Panenka, Agata Raczyk, Zuzanna Żol}
\title{Projekt PWI 2026 - Zespół 5 \\[0.5em] \Large Gra "Saper"}
\date{}

% początek dokumentu
\begin{document}
\maketitle
\thispagestyle{empty}

\section*{Wprowadzenie}
Projekt został zrealizowany w ramach przedmiotu \textit{Podstawowy Warsztat Informatyka} na Uniwersytecie Wrocławskim. Plik podsumowanie.tex został przygotowany w celu umożliwienia każdemu członkowi zespołu szczegółowego przedstawienia swojego wkładu w realizację zadania oraz opisania wykonanych prac.

\section{Karol Bednarek}
W ramach projektu byłem odpowiedzialny za funkcjonalność i wyświetlenie planszy oraz pomagałem przy tworzeniu menu, w tym głównie przy wybieraniu poziomów trudności gry. Zajmowałem się wizualnym wyświetleniem planszy, wszystkich symboli, instrukcji oraz zegara. Zintegrowałem funkcje generowania planszy i sterowania z interfejsem, uzyskując płynną rozgrywkę. Do tego napisałem fragment kodu, który zakończa grę wygraną lub przegraną i wyświetla komunikat. Napisałem również funkcję zapewniającą użytkownika, że jego pierwsze odkrycie pole będzie pustym polem. 

Jako źródła pomocnicze w pisaniu kodu używałem kursu internetowego o bibliotece Curses (ten sam co Agata Raczyk) oraz LLM Gemini do weryfikacji fragmentów kodu i debuggowania.

\section{Kinga Błaszkiewicz}
Mój wkład w projekt:
\begin{itemize}
    \item \textbf{Projekt i implementacja architektury:} Projektowałam oraz napisałam klasę planszy. Proponowałam wygląd klasy oraz funkcje, które na nią wpływają. Dbałam o spójność klasy i jej funkcji z resztą kodu.
    \item \textbf{Mechanika rozgrywki:} Zaimplementowałam moduł sterowania po planszy, funkcję zarządzania flagami (dodawanie/usuwanie) oraz sprawdzanie warunków zwycięstwa .
    \item \textbf{Zarządzanie danymi:} Odpowiadałam za inicjalizację obiektu planszy oraz integrację funkcji zapisu i wczytywania stanu gry.
\end{itemize}

\section{Jakub Chytroń}
Pracowałem nad generowaniem i obsługą planszy podczas rozgrywki. Zaprojektowałem oraz zaimplementowałem klasę plansza i metody obsługujące ruchy gracza, generowanie planszy oraz rozpoczęcie/zakończenie rozgrywki. Zaprojektowałem i zaprogramowałem rekurencyjne odsłanianie pól.

\section{Jerzy Panenka}

\textbf{Zadania za które byłem odpowiedzialny}
\begin{itemize}
    \item \textbf{Komunikacja z użytkownikiem:} Funkcje pobierające dane od użytkownika podczas tworzenia konta, logowania, oraz przekazujące dane do funkcji napisanych przez Zuzannę Żol.
    \item \textbf{Ranking: wyświetlanie:} Funkcja "wyswietl_ranking()" wyświetlająca najlepsze wymiki zalogowanych użytkowników zapisane za pomocą funkcji napisanych przez Zuzannę Żol.
    \item \textbf{Inne:} Integracja funkcji wczytywania gry, usuwania konta oraz zapisywania gry.
\end{itemize}

\textbf{Źródła:} Kurs biblioteki curses (ten sam co Karol Bednarek oraz Agata Raczyk), LLM Claude do debuggingu.

\section{Agata Raczyk}

\vspace{0.3cm}

Moimi głównymi zadaniami w projekcie była praca nad warstwą prezentacji programu (plik \texttt{wyswietlanie.py}) oraz dbałość o kompletność dokumentacji.\\

\textbf{Realizowane zadania programistyczne:}
\begin{itemize}
    \item \textbf{Menu główne:} Implementacja kompletnego systemu sterowania oraz interaktywnego menu startowego wraz z obsługą komunikatów dla użytkownika.
    \item \textbf{Interfejs rozgrywki:} Wdrożenie licznika flag oraz sformatowanie zegara gry do postaci \texttt{00:00:00}. 
    \\Dodanie funkcjonalności migania aktywnego pola, co ułatwia graczowi lokalizację na planszy.
    \item Opracowanie i wyświetlenie zasad gry wewnątrz programu.
\end{itemize}

\bigskip

\textbf{Zadania organizacyjne i dokumentacja:}
\begin{itemize}
    \item Przygotowanie środowiska pracy nad raportem (stworzenie szablonu \texttt{podsumowanie.tex}).
    \item Opracowanie pliku \texttt{README.md} (instrukcja i opis projektu).
    \item Pełnienie funkcji osoby kontaktowej z opiekunem projektu.
\end{itemize}

\bigskip % To robi odstęp od poprzedniego tekstu

\textbf{Źródła i materiały pomocnicze:}
\begin{itemize}
    \item \textbf{Kurs wideo:} \textit{Python Curses Tutorial - Make GOOD Looking Terminal Apps} (kanał Tech With Tim) \\
    \url{https://www.youtube.com/watch?v=Db4oc8qc9RU}
    
    \item \textbf{Strony internetowe:}
    \begin{itemize}
        \item \url{https://docs.python.org}
        \item \url{https://stackoverflow.com}
        \item \url{https://www.markdownguide.org/}
    \end{itemize}
    
    \item \textbf{LLM:} Google Gemini – konsultacje w zakresie redakcji dokumentacji oraz optymalizacji fragmentów kodu.
\end{itemize}

\section{Zuzanna Żol}
\begin{itemize}
    \item Samodzielnie zaprojektowałam i zaimplementowałam system obsługujący konta użytkowników, zapis oraz wczytywanie rozgrywki a także ranking wyników (działający za pomocą plików .json)
    \item Pomagałam również przy łączeniu back-endu dotyczącego systemu kont z front-endem
\end{itemize}

Do pracy używałam informacji ze strony \url{https://www.geeksforgeeks.org/python/python-json/} oraz ChatGPT (odpowiednie fragmenty są oznaczone w kodzie)

\section*{Czego nie udało się wykonać}
Oby ten paragraf pozostał pusty ...

\end{document}
